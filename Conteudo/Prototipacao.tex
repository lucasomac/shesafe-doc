\chapter{Desenvolvimento da aplicação}
\section{Avaliação com usuário}
A avaliação com usuário constitui-se como um paradigma metodológico fundamental no campo da Interação Humano-Computador (IHC), caracterizado pela participação direta de usuários reais no processo de análise e validação de sistemas interativos \cite{dix2003human}. Esta abordagem fundamenta-se no princípio de que a qualidade de um sistema deve ser mensurada a partir da perspectiva de quem efetivamente o utiliza, considerando suas necessidades, expectativas, limitações cognitivas e contextos de uso específicos \cite{preece2015interaction}.

Diferentemente dos métodos de inspeção realizados por especialistas, a avaliação com usuário oferece insights diretos sobre a experiência real de interação, revelando aspectos que podem não ser identificados através de análises teóricas ou heurísticas \cite{nielsen1994usability}. Esta metodologia permite a observação de comportamentos autênticos, identificação de estratégias de uso não previstas pelos designers, e compreensão das dificuldades reais enfrentadas pelos usuários em contextos naturais de utilização.

\subsection{Avaliação do usuário SheSafe}
As avaliações com usuário da aplicação SheSafe foram conduzidas entre os dias 24 e 28 de setembro de 2025, envolvendo participantes que executaram três tarefas específicas relacionadas às funcionalidades principais do sistema. Os testes foram realizados com a versão debug da aplicação Android, disponibilizada através de um link de download compartilhado.

Em relação ao cumprimento dos objetivos das tarefas, os resultados demonstraram 100\% de taxa de sucesso, com todas as participantes conseguindo completar as três atividades propostas: enviar um pedido de socorro pela primeira vez, cadastrar um novo contato seguro e alterar a mensagem padrão do pedido de ajuda. Não foram registrados erros durante a execução de nenhuma das tarefas, indicando que a interface apresenta boa intuitividade e que os fluxos de interação estão adequadamente projetados para facilitar a conclusão das ações pelos usuários.

A análise dos tempos de execução revelou performance satisfatória em todas as tarefas avaliadas. Para o envio do primeiro pedido de socorro, os tempos variaram entre 21 e 31 segundos, com média de 25,3 segundos. O cadastro de contatos seguros apresentou tempos mais consistentes, variando entre 13 e 17 segundos (média de 14,3 segundos). A alteração da mensagem padrão foi a tarefa executada mais rapidamente, com tempos entre 11 e 14 segundos (média de 12,3 segundos). Esta progressão temporal sugere que as funcionalidades mais críticas mantêm tempos de resposta adequados para situações de emergência, enquanto as funcionalidades de configuração são executadas de forma ainda mais eficiente.

No que se refere às impressões qualitativas, as avaliadoras expressaram percepções predominantemente positivas sobre diferentes aspectos da aplicação. A navegação foi consistentemente avaliada como adequada por todas as participantes, assim como o design visual e a usabilidade geral do sistema.
\begin{comment}
	Mônica Abreu destacou especificamente que "a interface é muito fácil e simples de aprender a utilizar", enquanto Wannielly Barbosa elogiou a "navegação fluida e sem problema para mudar de tela", caracterizando ainda a iniciativa como "ótima para mulheres".
\end{comment}

Contudo, emergiu um padrão consistente de feedback relacionado à ausência de confirmações explícitas do sistema após a execução de determinadas ações. Algumas das avaliadoras relataram especificamente a "falta de um retorno ao cadastrar o contato", enquanto uma outra observou que "ao alterar mensagem, só vi que alterou quando mudou na tela". Este feedback indica uma oportunidade de melhoria significativa no design de interação, particularmente na implementação de feedback imediato e explícito para ações críticas do usuário.

A convergência deste feedback específico sobre a ausência de confirmações do sistema sugere que esta deficiência pode impactar negativamente a confiança do usuário na aplicação, especialmente em um contexto de uso onde a certeza sobre a execução correta das ações é fundamental para a eficácia do sistema de segurança. A implementação de mensagens de confirmação, notificações toast, ou outros elementos de feedback visual imediato deveria ser considerada como prioridade para futuras iterações do desenvolvimento, visando aumentar a transparência do sistema e a confiança do usuário nas operações realizadas.L