\chapter{Prototipação}
\section{Protótipo}
O protótipo inicial pode acessado através do link: \href{https://www.figma.com/proto/ZOxt5eHuQt0RjhagaDpuXU/SheSafe?type=design&node-id=26-369&viewport=1892%2C1064%2C0.71&t=gZiNpzVt2mGhuMDV-0&scaling=min-zoom&starting-point-node-id=26%3A653}{SheSafe - Protótipo Inicial}

O protótipo inicial pode acessado através do link: \href{https://www.figma.com/proto/GALwTZKTsmvOVWX4JARmOB/SheSafe-Corrigido?node-id=26-653&viewport=2654%2C786%2C0.79&t=dkVCQTw83BbPw3KK-0&scaling=min-zoom&starting-point-node-id=26%3A653}{SheSafe - Protótipo Final}

Foi realizada uma cópia do projeto inicial para realizar as melhorias e alterações propostas pelas avaliações rasteira, heurística e de usabilidade.

\section{Melhorias Aplicadas}
Foram realizadas algumas mudanças no protótipo inicial até chegar no produto final. Tais melhorias foram realizadas com base nas avaliações aplicadas a público externo. Dentre elas estão:

\begin{itemize}
\item Tela 7 – alteração do título da tela para atender a corretamente ao que se propõe.
\item Tela de lista de contato – foi adicionado um menu de contexto com a opções de editar e excluir contato atendendo ao apontamento de não ser possível editar o contato.
\item Tela de listagem de contato – foi adicionado um dialog de confirmação de exclusão para atender ao apontamento de prevenção de exclusão acidental.
\item Tela 6 – foi alterado o ícone do menu de logoff para melhor exibir a intenção do menu.
\item Tela de alteração de mensagem – foi adicionada um feedback informando o sucesso da alteração da mensagem padrão do pedido de ajuda.
\item Campo de busca – adicionado hint explicativo para atender ao apontamento de não saber o que inserir no campo se busca.
\item Ícone de voltar – aplicada a redução do padrão de tamanho do ícone de voltar para atender ao apontamento recebido.
\item Botão de adicionar contato – foi introduzido um click longo para exibir um toltip explicando o que a ação do botão faz.
\item Menu de ações tela 6 – foi adicionado um toltip para ação de pressionar e segurar para os menus de ações da tela 6 atendendo ao apontamento de não saber exatamente o que cada menu faz. Com o toltip o menu passa a ter uma mensagem de contexto informado o que sua ação está encarregada.
\item Card de pedido enviado – aplicado estilo sublinhado no link do maps que contêm a localização do usuário atendendo ao apontamento sobre não reconhecer como link na tela o texto.
\item Dialogs – aplicado um pouco mais de peso no fundo dos botões de ação que ficam internos dentro dos dialogs para um melhor contraste.

\end{itemize}