\chapter{Estudo de viabilidade}\label{sec:EstuViab}

\section{Tipos de instrumento utilizados para a coleta de dados}
Foi utilizado para coleta de dados a abordagem de questionário com algumas perguntas distribuídas em quatro seções sendo elas idade/localidade, segurança, aplicativo SheSafe e comentários. 
A seção de idade e localidade continham perguntas relacionadas a idade dos usuários e a localidade, neste caso o estado, no qual eles vivem.
A seção de segurança abordou perguntas sobre a percepção do entrevistado em relação a questões como medo de sair na rua a noite, sobre já ter sofrido algum tipo de violência, entre outras perguntas que serão apresentadas na análise da coleta de dados.
A seção sobre o aplicativo SheSafe abordou perguntas quanto ao uso de um app que fosse capaz de ajudar aos entrevistados quando estivessem passando por uma situação adversa ou de risco. Também foram feitas perguntas sobre quais funcionalidades gostariam de ver em um aplicativo como este.
A seção de comentários ficou aberta para que os entrevistados pudessem expressar suas ideias e sugestões sobre a iniciativa do aplicativo SheSafe
Tal instrumento foi escolhido por julgar que alcançaria mais pessoas, pois seria disponibilizado através de redes sociais.

\section{Abordagem e período de coleta}
O questionário foi disponibilizado por meio de um post no LinkedIn e através de divulgação em grupos de aplicativos de mensagens instantâneas desde o dia 24 de fevereiro de 2024 até o dia 9 de março de 2024. No post foi informado que o questionário era voltado para pessoas do sexo feminino.

\section{Questões utilizadas para a coleta de dados}
\begin{itemize}
  \item Idade e estado onde reside
  \item Já deixou de ir a algum lugar com medo de não ser seguro para você pelo fato de ser mulher?
  \item Você se sente segura ao andar na rua?
  \item Do que você mais tem medo quando sai sozinha à noite?
  \item Você já foi vítima de algum tipo de violência ou assédio?
  \item Você usaria um aplicativo que informasse se um determinado local possui alto índice de crimes relacionados a mulher ou até mesmo o quão esse lugar é seguro para mulheres?
  \item Quais funcionalidades você gostaria de ver em um aplicativo como este?
  \item Se você tiver algum outro comentário ou sugestão, por favor, escreva aqui:
\end{itemize}

\section{Resultados Obtidos}
\subsection{Idade e Localidade}
A primeira pergunta realizada foi sobre a idade do usuário que com base no gráfico abaixo podemos ver que a predominância foi entre as pessoas mais jovens. Pessoas mais jovens costumam utilizar mais tecnologia como meio de se comunicar sendo assim o maior alcance a esse grupo. 

\begin{figure}[htbp]
  \begin{center}
  \includegraphics[width=1.0\linewidth]{images/distribuicao-idade.png}\\
  \end{center}
  \caption[Distribuição de faixa etária]{Distribuição de faixa etária}
  \label{fig:mapa-empatia=inicial}
  \legend{Fonte: Próprio Autor}
\end{figure}

A segunda pergunta foi quanto a localidade (estado) em que o entrevistado mora. Pelo gráfico abaixo podemos que ver maior parte dos entrevistados foram oriundos do estado de São Paulo. Embora a pesquisa não tenha conseguido coletar dados e pessoas de alguns estados, podemos ver que conseguimos abranger uma boa parte dos estados brasileiros.

\begin{figure}[htbp]
  \begin{center}
  \includegraphics[width=1.0\linewidth]{images/distribuicao-estados.png}\\
  \end{center}
  \caption[Distribuição de localização]{Distribuição de localização}
  \label{fig:mapa-empatia=inicial}
  \legend{Fonte: Próprio Autor}
\end{figure}

\subsection{Segurança}
A primeira pergunta relacionada a segurança foi quanto ao fato do entrevistado se privar de ir a lugares por conta de ser uma mulher. O gráfico abaixo nos mostra que infelizmente muitas mulheres ainda se privam de ir a determinados lugares por conta do medo do que pode acontecer a elas simplesmente por serem mulheres. 
\begin{figure}[h]
  \begin{center}
  \includegraphics[width=1.0\linewidth]{images/distribuicao-privacao-mulher.png}\\
  \end{center}
  \caption[Distribuição de privação de local]{Distribuição de privação de local}
  \label{fig:mapa-empatia=inicial}
  \legend{Fonte: Próprio Autor}
\end{figure}

A segunda pergunta foi sobre o entrevistado se sentir seguro ao andar na rua. O resultado abaixo nos mostra que majoritariamente as mulheres não se sentem seguras ao andar na rua. Mesmo quando não em sua totalidade, elas ainda sentem receio de andar na rua em determinados lugares. 
\begin{figure}[h]
  \begin{center}
  \includegraphics[width=1.0\linewidth]{images/distribuicao-seguranca-rua.png}\\
  \end{center}
  \caption[Distribuição de privação de local]{Distribuição de privação de local}
  \label{fig:mapa-empatia=inicial}
  \legend{Fonte: Próprio Autor}
\end{figure}

A terceira pergunta desta seção foi relacionada ao que a mulher sentia mais medo ao sair à noite sozinha. A pergunta poderia ter várias respostas escolhidas e os resultados abaixo nos mostram que a violência ainda continua sendo o principal medo reportado por elas.
\begin{figure}[h]
  \begin{center}
  \includegraphics[width=1.0\linewidth]{images/distribuicao-medo.png}\\
  \end{center}
  \caption[Distribuição de maior medo]{Distribuição de maior medo}
  \label{fig:mapa-empatia=inicial}
  \legend{Fonte: Próprio Autor}
\end{figure}

A última pergunta da seção de segurança foi um pouco mais direta quanto ao fato de o entrevistado já ter sofrido algum tipo de violência ou assédio. Mais uma vez majoritariamente podemos ver que infelizmente muitas mulheres já passaram por essas situações. 
\begin{figure}[h]
  \begin{center}
  \includegraphics[width=1.0\linewidth]{images/distribuicao-vitma.png}\\
  \end{center}
  \caption[Distribuição de vítima]{Distribuição de vítima}
  \label{fig:mapa-empatia=inicial}
  \legend{Fonte: Próprio Autor}
\end{figure}