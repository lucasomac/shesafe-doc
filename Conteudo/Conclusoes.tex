\chapter{Implementações futuras}

O desenvolvimento da aplicação SheSafe, embora tenha atingido seus objetivos principais de oferecer uma ferramenta de proteção e socorro com funcionalidades essenciais de geolocalização e comunicação de emergência através do seu MVP, apresenta diversas oportunidades de expansão e aprimoramento que podem aumentar significativamente sua eficácia e alcance. As propostas para trabalhos futuros fundamentam-se tanto nas limitações identificadas durante o desenvolvimento atual quanto na necessidade de ampliar a compatibilidade e funcionalidades do sistema para atender de forma mais abrangente às demandas dos usuários.

Uma das principais expansões propostas refere-se à implementação de autenticação através do sistema Sign In with Apple, funcionalidade que demandará a aquisição de uma licença anual do Apple Developer Program. Esta integração representaria um avanço significativo na facilidade de acesso e segurança da aplicação, uma vez que o sistema de autenticação da Apple oferece recursos avançados de privacidade, incluindo a possibilidade de ocultação do endereço de email real do usuário e autenticação biométrica integrada. A implementação desta funcionalidade seria particularmente relevante considerando que a proteção da privacidade constitui um aspecto crítico para aplicações voltadas à segurança pessoal, onde a confidencialidade das informações dos usuários pode ser determinante para sua proteção efetiva.

A integração de alertas através do WhatsApp representa outra vertente promissora para desenvolvimento futuro, considerando que esta plataforma de comunicação apresenta penetração massiva no mercado brasileiro e é amplamente utilizada por diferentes faixas etárias e perfis socioeconômicos. A implementação desta funcionalidade envolveria a utilização da WhatsApp Business API, permitindo que a aplicação envie mensagens automatizadas de socorro com informações de geolocalização diretamente para os contatos predefinidos pelo usuário. Esta integração ofereceria vantagens significativas em termos de familiaridade da interface para os receptores dos alertas, maior confiabilidade na entrega das mensagens devido à infraestrutura robusta do WhatsApp, e possibilidade de incluir elementos multimídia como localização em tempo real e mensagens de voz gravadas durante situações de emergência.

O desenvolvimento de funcionalidades para importação automática de contatos a partir da lista de contatos do dispositivo móvel constituiria uma melhoria substancial na experiência do usuário, eliminando a necessidade de digitação manual das informações de contatos seguros. Esta implementação demandaria o gerenciamento cuidadoso de permissões do sistema operacional Android, garantindo que o acesso aos contatos seja solicitado de forma transparente e utilizado exclusivamente para os propósitos declarados da aplicação. A funcionalidade poderia incluir recursos de filtragem inteligente, permitindo que o usuário selecione facilmente contatos específicos de sua lista, bem como validação automática de números de telefone para garantir a funcionalidade adequada do sistema de alertas.

A expansão da aplicação SheSafe para o ecossistema iOS representa uma oportunidade estratégica fundamental para ampliar o alcance da ferramenta de proteção. O desenvolvimento para iOS envolveria não apenas a adaptação do código existente para as especificidades da plataforma Apple, mas também a implementação de funcionalidades nativas do sistema operacional, como integração com o Sistema de Notificação de Emergência, compatibilidade com o Apple Watch para acionamento discreto de alertas, e aproveitamento de recursos específicos como o Emergency SOS já integrado aos dispositivos Apple. Esta expansão multiplataforma permitiria que a aplicação atendesse a um espectro mais amplo de usuários, independentemente da plataforma móvel utilizada, contribuindo para o objetivo social mais amplo de proteção contra violência de gênero.

As propostas de trabalhos futuros também deveriam contemplar a publicação oficial da aplicação SheSafe na Google Play Store, representando uma etapa crucial para ampliar significativamente a base de usuários e estabelecer um canal de distribuição profissional e confiável. A disponibilização através da loja oficial do Android ofereceria múltiplas vantagens estratégicas, incluindo maior visibilidade para o público-alvo, credibilidade institucional através do processo de revisão da Google, e acesso a ferramentas analíticas avançadas para monitoramento do desempenho e comportamento dos usuários. Esta iniciativa permitiria a coleta sistemática de feedback através do sistema de avaliações da Play Store, fornecendo insights valiosos sobre a experiência real dos usuários em contextos diversos e identificando padrões de uso que não emergem em testes controlados com grupos reduzidos.

A expansão da base de usuários através da publicação na Play Store facilitaria a implementação de metodologias de pesquisa em escala, incluindo análise de métricas de engajamento, identificação de pontos de abandono na jornada do usuário, e compreensão dos contextos de uso mais frequentes da aplicação. Estes dados seriam fundamentais para orientar iterações futuras do desenvolvimento, permitindo decisões baseadas em evidências sobre priorização de funcionalidades, otimizações de interface, e adaptações para diferentes perfis demográficos de usuários. Adicionalmente, a maior base de usuários possibilitaria a realização de testes A/B para validação de diferentes abordagens de design e funcionalidade, contribuindo para o aprimoramento contínuo da eficácia da ferramenta de proteção.
\begin{comment}
A realização de estudos empíricos mais extensivos com usuários finais também representaria uma vertente importante para trabalhos futuros, incluindo testes de usabilidade em condições simuladas de estresse, avaliações de acessibilidade com usuários com diferentes tipos de deficiência, e estudos longitudinais para avaliar a eficácia real da aplicação em contextos de uso prolongado. Adicionalmente, seria relevante investigar a integração com órgãos públicos de segurança e organizações de proteção às mulheres, criando canais diretos para encaminhamento de alertas quando apropriado e em conformidade com as preferências e consentimento dos usuários.
\end{comment}